\documentclass[dutch]{article}
\usepackage[a4paper,text={20cm,29cm},centering]{geometry}
\usepackage[table]{xcolor}
\usepackage{amsmath, mathtools}

\setlength{\tabcolsep}{10pt}
\renewcommand{\arraystretch}{2}

\newcommand{\note}[1]{\textcolor{gray}{\tiny (#1)}}
% TODO: more coefficients!!!
\begin{document}
\centering
{\rowcolors{3}{gray!10}{white}
    \begin{tabular}{ l | l l l l}
        \hline
        \multicolumn{5}{c}{\huge{Waarden en constanten}} \\
        \hline
        Naam & symbool & waarde & eenheid (SI) & alternatieve eenheid \\
        \hline
        elementaire lading & $e$ & $1.60217\cdot 10^{-19}$ & C \note{Coulomb} & $4.803205$ Fr
        \note{franklins}\\
        Gravietatieconstante & $G$ & $6.67259\cdot 10^{-11}$ & $\frac{\text{m}^3}{\text{kg s}^2} =
        \frac{\text{N m}^2}{\text{kg}^2}$ & \\
        constante van Planck & $h$ & $6.62607\cdot 10^{-34}$ & Js = $\frac{\text{kg m}^2}{s}$ &
        $4.13566\cdot10^{-15}eV/s$ \\
        constante van Dirac & $\hbar=h/2\pi$ & $1,0545727\cdot10^{-34}$ & Js & \\
        lichtsnelheid in vacu\"um & $c$ & $2.99792\cdot10^8$ & m/s & $6.706\cdot10^8$ mph\\
        permittiviteit van het vacu\"um & $\epsilon_0$ & $8.85418\cdot10^{-12}$ & F/m
        \note{Farad per meter} & \\
        permeabiliteit van het vacu\"um & $\mu_0$ & $4\pi\cdot10{-7}$ & H/m \note{Hanry per meter} &
        \\
        fijnstructuurconstante & $\alpha = \frac{e^2}{2hc\epsilon_0}$ & $\approx 1/137$ & & \\ %TODO
        bohrmagneton & $\mu_{B}=e\hbar/2m_{e}$ & $9,2741\cdot10^{-24}$ & Am$^2$ & $0.46686 \text{ cm}^{-1}$/T \\
        bohrstraal & $a_0$ & $0,52918$ & & \\
        rydbergconstante & $Ry$ & 13,595 & eV & \\
        comptongolflengte elektron & $\lambda_{Ce}=h/m_{e} c$ & $2,2463\cdot10^{-12}$
        & m & \\
        comptongolflengte proton & $\lambda_{Cp}=h/m_{p}c$ & $1,3214\cdot10^{-15}$
        & m & \\
        gereduceerde massa H-atoom & $\mu_{H}$ & $9,10457\cdot10^{-31}$ & kg & \\
        constante van Stefan-Boltzmann
        &$\sigma$ & $5,67032\cdot10^{-8}$ & Wm$^{-2}$K$^{-4}$ & \\
        constante van Wien & $k_{\rm W}$ & $2,8978\cdot10^{-3}$ & mK & \\
        \hline
        gasconstante & $R$ & $8.31441$ &J$\cdot$mol$^{-1}\cdot$K$^{-1}$ & \\
        getal van Avogadro & $N_{A}$ & $6.02213\cdot10^{23}$&mol$^{-1}$ & \\
        constante van Boltzmann & $k=R/N_{A}$ & $1.38065\cdot10^{-23}$ & J/K & \\
        \hline
        massa van het elektron & $m_{
            e}$ & $9.10938\cdot10^{-31}$ & kg & \\
        massa van het proton & $m_{p}$ & $1.67262\cdot10^{-27}$ & kg & \\
        massa van het neutron & $m_{n}$ & $1.67495\cdot10^{-27}$ & kg & \\
        elementaire massaeenheid & $m_{
            u}=\frac{1}{12}m(^{12}_{~6}$C)&$1.66056\cdot10^{-27}$ & kg & \\
        kernmagneton & $\mu_{N}$ & $5.0508\cdot10^{-27}$ & J/T\\
        \hline
        diameter van de Zon & $D_\odot$ & $1392\cdot10^6$ & m & \\
        massa van de Zon & $M_\odot$ & $1.989\cdot10^{30}$ & kg & \\
        straal van de Aarde & $R_{A}$ & $6.378\cdot10^6$ & m & \\
        massa van de Aarde & $M_{A}$ & $5.976\cdot10^{24}$ & kg & \\
        astronomische eenheid & AE & $1.49597\cdot10^{11}$ & m & \\
        lichtjaar & lj & $9.4605\cdot10^{15}$ & m & \\
        parsec & pc & $3.0857\cdot10^{16}$ & m & \\
        constante van Hubble & $H$ & $\approx(75\pm25)$ & km$\cdot$s$^{-1}\cdot$Mpc$^{-1}$
        & \\
        constante van de liefde & $<3$ & jij en ik & Kouznetsov en Deceuninck & \\
        \hline
    \end{tabular}
}

\end{document}
